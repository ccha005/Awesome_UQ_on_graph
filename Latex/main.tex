%%
%% This is file `sample-manuscript.tex',
%% generated with the docstrip utility.
%%
%% The original source files were:
%%
%% samples.dtx  (with options: `manuscript')
%% 
%% IMPORTANT NOTICE:
%% 
%% For the copyright see the source file.
%% 
%% Any modified versions of this file must be renamed
%% with new filenames distinct from sample-manuscript.tex.
%% 
%% For distribution of the original source see the terms
%% for copying and modification in the file samples.dtx.
%% 
%% This generated file may be distributed as long as the
%% original source files, as listed above, are part of the
%% same distribution. (The sources need not necessarily be
%% in the same archive or directory.)
%%
%%
%% Commands for TeXCount
%TC:macro \cite [option:text,text]
%TC:macro \citep [option:text,text]
%TC:macro \citet [option:text,text]
%TC:envir table 0 1
%TC:envir table* 0 1
%TC:envir tabular [ignore] word
%TC:envir displaymath 0 word
%TC:envir math 0 word
%TC:envir comment 0 0
%%
%%
%% The first command in your LaTeX source must be the \documentclass
%% command.
%%
%% For submission and review of your manuscript please change the
%% command to \documentclass[manuscript, screen, review]{acmart}.
%%
%% When submitting camera ready or to TAPS, please change the command
%% to \documentclass[sigconf]{acmart} or whichever template is required
%% for your publication.
%%
%%
\documentclass[manuscript,screen,review]{acmart}

%%
%% \BibTeX command to typeset BibTeX logo in the docs
\AtBeginDocument{%
  \providecommand\BibTeX{{%
    Bib\TeX}}}

%% Rights management information.  This information is sent to you
%% when you complete the rights form.  These commands have SAMPLE
%% values in them; it is your responsibility as an author to replace
%% the commands and values with those provided to you when you
%% complete the rights form.
\setcopyright{acmcopyright}
\copyrightyear{2018}
\acmYear{2018}
\acmDOI{XXXXXXX.XXXXXXX}

%% These commands are for a PROCEEDINGS abstract or paper.
\acmConference[Conference acronym 'XX]{Make sure to enter the correct
  conference title from your rights confirmation emai}{June 03--05,
  2018}{Woodstock, NY}
%%
%%  Uncomment \acmBooktitle if the title of the proceedings is different
%%  from ``Proceedings of ...''!
%%
%%\acmBooktitle{Woodstock '18: ACM Symposium on Neural Gaze Detection,
%%  June 03--05, 2018, Woodstock, NY}
\acmPrice{15.00}
\acmISBN{978-1-4503-XXXX-X/18/06}


%%
%% Submission ID.
%% Use this when submitting an article to a sponsored event. You'll
%% receive a unique submission ID from the organizers
%% of the event, and this ID should be used as the parameter to this command.
%%\acmSubmissionID{123-A56-BU3}

%%
%% For managing citations, it is recommended to use bibliography
%% files in BibTeX format.
%%
%% You can then either use BibTeX with the ACM-Reference-Format style,
%% or BibLaTeX with the acmnumeric or acmauthoryear sytles, that include
%% support for advanced citation of software artefact from the
%% biblatex-software package, also separately available on CTAN.
%%
%% Look at the sample-*-biblatex.tex files for templates showcasing
%% the biblatex styles.
%%

%%
%% The majority of ACM publications use numbered citations and
%% references.  The command \citestyle{authoryear} switches to the
%% "author year" style.
%%
%% If you are preparing content for an event
%% sponsored by ACM SIGGRAPH, you must use the "author year" style of
%% citations and references.
%% Uncommenting
%% the next command will enable that style.
%%\citestyle{acmauthoryear}


%%
%% end of the preamble, start of the body of the document source.
\begin{document}

%%
%% The "title" command has an optional parameter,
%% allowing the author to define a "short title" to be used in page headers.
\title{Uncertainty Quantification on Graph Learning: A Survey}

%%
%% The "author" command and its associated commands are used to define
%% the authors and their affiliations.
%% Of note is the shared affiliation of the first two authors, and the
%% "authornote" and "authornotemark" commands
%% used to denote shared contribution to the research.
%\author{Ben Trovato}
%\authornote{Both authors contributed equally to this research.}
%\email{trovato@corporation.com}
%\orcid{1234-5678-9012}
%\author{G.K.M. Tobin}
%\authornotemark[1]
%\email{webmaster@marysville-ohio.com}
%\affiliation{%
%  \institution{Institute for Clarity in Documentation}
%  \streetaddress{P.O. Box 1212}
%  \city{Dublin}
%  \state{Ohio}
%  \country{USA}
%  \postcode{43017-6221}
%}

%\author{Lars Th{\o}rv{\"a}ld}
%\affiliation{%
%  \institution{The Th{\o}rv{\"a}ld Group}
%  \streetaddress{1 Th{\o}rv{\"a}ld Circle}
%  \city{Hekla}
%  \country{Iceland}}
%\email{larst@affiliation.org}

%\author{Valerie B\'eranger}
%\affiliation{%
%  \institution{Inria Paris-Rocquencourt}
%  \city{Rocquencourt}
%  \country{France}
%}
%
%\author{Aparna Patel}
%\affiliation{%
% \institution{Rajiv Gandhi University}
% \streetaddress{Rono-Hills}
% \city{Doimukh}
% \state{Arunachal Pradesh}
% \country{India}}
%
%\author{Huifen Chan}
%\affiliation{%
%  \institution{Tsinghua University}
%  \streetaddress{30 Shuangqing Rd}
%  \city{Haidian Qu}
%  \state{Beijing Shi}
%  \country{China}}
%
%\author{Charles Palmer}
%\affiliation{%
%  \institution{Palmer Research Laboratories}
%  \streetaddress{8600 Datapoint Drive}
%  \city{San Antonio}
%  \state{Texas}
%  \country{USA}
%  \postcode{78229}}
%\email{cpalmer@prl.com}
%
%\author{John Smith}
%\affiliation{%
%  \institution{The Th{\o}rv{\"a}ld Group}
%  \streetaddress{1 Th{\o}rv{\"a}ld Circle}
%  \city{Hekla}
%  \country{Iceland}}
%\email{jsmith@affiliation.org}
%
%\author{Julius P. Kumquat}
%\affiliation{%
%  \institution{The Kumquat Consortium}
%  \city{New York}
%  \country{USA}}
%\email{jpkumquat@consortium.net}

%%
%% By default, the full list of authors will be used in the page
%% headers. Often, this list is too long, and will overlap
%% other information printed in the page headers. This command allows
%% the author to define a more concise list
%% of authors' names for this purpose.
\renewcommand{\shortauthors}{Trovato et al.}

%%
%% The abstract is a short summary of the work to be presented in the
%% article.
\begin{abstract}

\end{abstract}

%%
%% The code below is generated by the tool at http://dl.acm.org/ccs.cfm.
%% Please copy and paste the code instead of the example below.
%%
\begin{CCSXML}
<ccs2012>
 <concept>
  <concept_id>00000000.0000000.0000000</concept_id>
  <concept_desc>Do Not Use This Code, Generate the Correct Terms for Your Paper</concept_desc>
  <concept_significance>500</concept_significance>
 </concept>
 <concept>
  <concept_id>00000000.00000000.00000000</concept_id>
  <concept_desc>Do Not Use This Code, Generate the Correct Terms for Your Paper</concept_desc>
  <concept_significance>300</concept_significance>
 </concept>
 <concept>
  <concept_id>00000000.00000000.00000000</concept_id>
  <concept_desc>Do Not Use This Code, Generate the Correct Terms for Your Paper</concept_desc>
  <concept_significance>100</concept_significance>
 </concept>
 <concept>
  <concept_id>00000000.00000000.00000000</concept_id>
  <concept_desc>Do Not Use This Code, Generate the Correct Terms for Your Paper</concept_desc>
  <concept_significance>100</concept_significance>
 </concept>
</ccs2012>
\end{CCSXML}

\ccsdesc[500]{Do Not Use This Code~Generate the Correct Terms for Your Paper}
\ccsdesc[300]{Do Not Use This Code~Generate the Correct Terms for Your Paper}
\ccsdesc{Do Not Use This Code~Generate the Correct Terms for Your Paper}
\ccsdesc[100]{Do Not Use This Code~Generate the Correct Terms for Your Paper}

%%
%% Keywords. The author(s) should pick words that accurately describe
%% the work being presented. Separate the keywords with commas.
\keywords{Do, Not, Us, This, Code, Put, the, Correct, Terms, for,
  Your, Paper}

\received{20 February 2007}
\received[revised]{12 March 2009}
\received[accepted]{5 June 2009}

%%
%% This command processes the author and affiliation and title
%% information and builds the first part of the formatted document.
\maketitle

\section{Introduction}
\section{Preliminaries}
\label{sec:preliminaries}
In this section, we briefly introduce two key families of models for graphs, the Probabilistic Graphical Model and the Graph Neural Network.

\subsection{Probabilistic Graphical Model}
A probabilistic graphical model (PGM) \cite{bishop2006pattern,jordan2004graphical} is widely used to represent the probability distribution of random variables, whose relations can be depicted in a graph.
PGMs aim to minimize the cost of establishing compatible dependency relationships among all the variables. 
There are two distinct categories of PGMs, depending on whether the graph's edges are directed or undirected. 
Specifically, a Bayesian Network is employed for modeling a directed graph, where edges represent causality. 
Conversely, a Markov Random Field (MRF) is utilized to model an undirected graph, where edges signify the correlation between nodes. 
In an MRF, each node is conditionally independent of all other nodes, except for its immediate (e.g., 1-hop) neighbors.
% In this paper, we will focus on MRF.

Formally, 
we denote a graph $G=(V,E)$ consisting of a set of $n$ random variables $V=\{ X_1, \dots, X_n \}$ and $E$ implies the relationship between any two variables. 
Each variable $X_i \in V$ may take values in $\{1,\dots, C\}$ where $C$ is the number of classes.
% given a set of $n$ random variables $V=\{ X_1, \dots, X_n \}$, each taking values in $\{1,\dots, C\}$ where $C$ is the number of classes, 
An MRF factories the joint distribution $P(V)$ as 
\begin{equation}
\label{eq:pgm_joint}
    P(V)=\frac{1}{Z}\prod_{X_i\in V} \phi(X_i=x_i) \prod_{X_j\in \mathcal{N}(X_i)}\psi(X_i=x_i, X_j=x_j),
\end{equation}
where $Z$ normalizes the product to a probability distribution.
$\mathcal{N}(X_i)$ represents the neighbors of $X_i$ within the graph $G$.
$\phi(X_i=x_i)$ denotes the prior distribution of $X_i$ taking on the value $x_i$, independent of other variables in the graph.
The compatibility $\psi(X_i=x_i, X_j=x_j)$ encodes the likelihood of the pair
$(X_i, X_j)$ jointly taking the value $(x_i, x_j)$,
and capture the dependencies between variables.
The marginal distribution $b(X_i)$, also known as belief, for any $X_i \in V$ is naturally given by
\begin{equation}
\label{eq:pgm_margin}
    b(X_i=x_i) = \sum_{X_1} \dots \sum_{X_{i-1}} \sum_{X_{i+1}} \dots \sum_{X_n} P(X_1, X_2, \dots, X_i=x_i, \dots, X_n).
\end{equation}

However, computing $b(X_i=x_i)$ for any node $X_i \in V$ by Eq. (\ref{eq:pgm_margin}) is exponential complexity in the worst case. 
We show that Belief Propagation (BP) \cite{bishop2006pattern} infers the beliefs much more efficiently. 
First, the message $m_{i\to j}(X_j=x_j)$ from $X_i$ to $X_j$ is defined by
\begin{equation}
\label{eq:pgm_sum_prod}
\frac{1}{Z_j}\sum_{X_i=x_i}
\left[\psi(X_i=x_i, X_j=x_j)\phi(X_i=x_i) \prod_{k\in \mathcal{N}(X_i)\setminus \{X_j\}} m_{k\to i}(X_i=x_i)\right],
\end{equation}
where $Z_j$ is a normalization factor so that $m_{i\to j}$ is a probability distribution of $X_j$.
The messages in both directions on all edges $(X_i, X_j) \in E$ will be updated iteratively. 
As the message updating converges (guaranteed when $G$ is acyclic~\cite{pearl1988probabilistic}).
The belief $b(X_i=x_i)$ is given by
\begin{equation}
\label{eq:pgm_belief}
    b(X_i=x_i) \propto \phi(X_i=x_i)\prod_{X_j\in \mathcal{N}(X_i)} m_{j\to i}(X_i=x_i).
\end{equation}
For simplicity, we will omit $X_i$ in $\phi(X_i=x_i)$ and use $\phi(x_i)$, and similarly for $\psi(x_i,x_j)$, $m_{j\to i}(x_i)$ and $b(x_i)$ if there is no ambiguity.

\subsection{Graph Neural Networks}
Graph Neural Network (GNN) is a deep learning-based method that 
\section{Uncertainty Quantification on Graphs}
\subsection{Uncertainty Quantification in Probabilitic Graphical Models}

\subsection{Uncertainty Quantification in Graph Neural Networks}

\documentclass{article}
\usepackage{graphicx} % Required for inserting images

\title{UQML}
\author{Rui Xu}
\date{November 2023}



\maketitle

\section{Reviews}
\subsection{Calibrating Uncertainty Models for Steering Angle Estimation - 2019 IEEE}
This paper reviews three main kinds of UQ methods in machine learning: Monte Carlo Dropout, Bootstrap Model, and Gaussian Mixture Model. Evaluation metrics are also introduced. Prediction accuracy, usually chosen as RMSE or MAE, is used to see if the predicted value is close to the training data. Calibration, on the other hand, is to measure if the predicted distribution is similar to the training data distribution.  
\subsubsection{Monte Carlo Dropout}
\begin{figure}[ht]
\includegraphics[width=3in]{MC dropout.png}
\end{figure}
\noindent To construct the dropout variant based on the baseline model, dropout layers are added after every fully connected layer but the last. Dropout is implemented by sampling a Bernoulli random variable with probability p. Neuron k in layer l thus gets dropped with probability $z^{l;k}$  $\sim$ Bernoulli(p).\\

\noindent Inference for an unknown sample x is performed by doing multiple stochastic forward passes and averaging the result. The randomness comes from generating new dropout weight matrices $\hat{W}_t$ from the Bernoulli distribution for every forward pass t. A forward pass with network weights $\hat{W}_t$ is denoted as $f ^{\hat{W}_t}(x)$. The predicted mean steering angle can then be calculated as
\[ E[\hat{y}]=\frac{1}{T}\sum_{t=1}^{T}\hat{y}_t=\frac{1}{T}\sum_{t=1}^{T}f ^{\hat{W}_t}(x) \]
We can also get the predictive variance of the stochastic forward passes, seen as our uncertainty estimate, by calculating the variance over all T results plus the inverse model
precision $\tau^{-1}$ as
\[Var[\hat{y}]=\tau^{-1}+\frac{1}{T}\sum_{t=1}^{T}(f ^{\hat{W}_t}(x))^2-(E[\hat{y}])^2\]
The model precision $\tau^{-1}$ can be interpreted as label noise and thus representing the aleatoric uncertainty. Being a fixed constant it implies homoscedastic aleatoric uncertainty. We omit the model precision and use the variance of the outputs as predictive uncertainty.
\subsubsection{Bootstrap Model}
We would have to train models separately and make inferences in those networks K in parallel, we rely on a simplification instead: multiple models are fused into one model by sharing the visual encoder network.\\

\noindent The process of sampling K subsets of the training data for each subnetwork is done by generating a mask for every sample indicating which head it is passed to during training.\\

\noindent This ensures that every head sees only a subset of the whole training data. The data subsampling is realized by generating a binary mask $\{0, 1\}^K$ for every training sample indicating whether it is seen by head $k \in [1, K]$.\\

\begin{figure}[ht]
\includegraphics[width=3in]{Bootstrap.png}
\end{figure}

\noindent For sample t and head k, this results in variable $m^{k}_t\sim$ Bernoulli(p) with p = 0:5 specifying whether a head is trained on a certain sample.\\

\noindent The loss thus is defined as
\[L(\theta)=\sum_{i=1}^{N}\sum_{k=1}^{K}m^k_i\frac{1}{2}||y_i-\hat{y}_i||^2\]

\noindent Inference is performed by forward-propagation through all heads resulting in K outputs. Note that only a single pass is required to get K outputs, instead of having to do K passes as in the dropout model. $f^k(x)$ is the prediction of head $k$ for sample $x$. 
\[E[\hat{y}]=\frac{1}{K}\sum_{k=1}^{K}f^k(x)\]
\[Var[\hat{y}]=\frac{1}{K}\sum_{k=1}^{K}(f^k(x))^2-(E[\hat{y}])^{2}\]

\noindent Bootstrap model can not get aleatoric and epistemic uncertainty separated. Instead, it only yields the predictive variance which is used as predictive uncertainty. The inference step is computationally more expensive than inference in the baseline model as there are K more computations in the fully connected layer.
\subsubsection{Gaussian Mixture Model}
A Gaussian Mixture Model is composed of multiple Gaussians combined in a weighted sum. The parameters of a GMM are $\theta = \pi_i; \mu_i; \sigma^2_i; i \in [1, K]$ for $K$ mixtures.
\begin{center}
\includegraphics[width=2in]{GMM.png}
\end{center}
\noindent The mixture model is trained using the negative log-likelihood as a loss. A small constant 
$\epsilon$ = 1e - 6 is added to the argument of the logarithm for numerical stability.
\[L(\theta)=-\frac{1}{N}\sum_{i=1}^{N}\log(p(y_i|x_i))=\-\frac{1}{N}\sum_{i=1}^{N}\log(\sum_{j=1}^{K}\pi_j^{(i)}N(y_i|\mu_j^{(i)},\sigma_j^{2(i)})+\epsilon)\]
\noindent All mixture components are combined in a weighted sum and we get the following results for the expected value:
\[E[\hat{y}]=\sum_{j=1}^{K}\pi_j(x)\mu_j(x)\]
\noindent The total variance, in our context, called the predicted variance which is interpreted as predictive uncertainty, decomposes into the weighted sum of the variances and the weighted variances of the means,
\[Var[\hat{y}]=\sum_{j=1}^{K}\pi_j(x)\sigma_j^{2}(x)+\sum_{j=1}^{K}\pi_j(x)||\mu_j(x)-\sum_{k=1}^{K}\pi_k(x)\mu_k(x)||^{2}\]
\noindent The second term is also referred to as explained variance or epistemic uncertainty as it vanishes with more data. A decreasing aleatoric uncertainty can be interpreted as a decreasing variance of the mixture components. In turn, intuitively, a decreasing epistemic uncertainty equals to the means of the mixture components getting closer. However, this is only the case as long as there are no ambiguous situations as would be the case at intersections with multiple possible routes.


\subsubsection{Prediction Accuracy}
The prediction accuracy is usually chosen as root mean square error(RMSE) or mean absolute error(MAE) to measure the distance between the predicted values and true values. 
\subsubsection{Calibration}
\noindent Intuitively, this means that the uncertainty estimations have to be a true probability and should reflect the true likelihood.
\begin{center}
\includegraphics[width=2.75in]{Calibration.png}
\end{center}
\noindent First, a z\% confidence interval is computed for each sample in the data set based on the predicted mean and variance. The confidence interval (CI) specifies the lower and upper bound arranged symmetrically around the mean, containing z\% of the given Gaussian distribution mass.




%\section{Acknowledgments}

\bibliographystyle{ACM-Reference-Format}
\bibliography{sample-base}

\appendix

\end{document}
\endinput
%%
%% End of file `main.tex'.
