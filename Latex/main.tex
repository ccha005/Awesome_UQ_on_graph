%%
%% This is file `sample-manuscript.tex',
%% generated with the docstrip utility.
%%
%% The original source files were:
%%
%% samples.dtx  (with options: `manuscript')
%% 
%% IMPORTANT NOTICE:
%% 
%% For the copyright see the source file.
%% 
%% Any modified versions of this file must be renamed
%% with new filenames distinct from sample-manuscript.tex.
%% 
%% For distribution of the original source see the terms
%% for copying and modification in the file samples.dtx.
%% 
%% This generated file may be distributed as long as the
%% original source files, as listed above, are part of the
%% same distribution. (The sources need not necessarily be
%% in the same archive or directory.)
%%
%%
%% Commands for TeXCount
%TC:macro \cite [option:text,text]
%TC:macro \citep [option:text,text]
%TC:macro \citet [option:text,text]
%TC:envir table 0 1
%TC:envir table* 0 1
%TC:envir tabular [ignore] word
%TC:envir displaymath 0 word
%TC:envir math 0 word
%TC:envir comment 0 0
%%
%%
%% The first command in your LaTeX source must be the \documentclass
%% command.
%%
%% For submission and review of your manuscript please change the
%% command to \documentclass[manuscript, screen, review]{acmart}.
%%
%% When submitting camera ready or to TAPS, please change the command
%% to \documentclass[sigconf]{acmart} or whichever template is required
%% for your publication.
%%
%%
\documentclass[manuscript,screen,review]{acmart}

%%
%% \BibTeX command to typeset BibTeX logo in the docs
\AtBeginDocument{%
  \providecommand\BibTeX{{%
    Bib\TeX}}}

%% Rights management information.  This information is sent to you
%% when you complete the rights form.  These commands have SAMPLE
%% values in them; it is your responsibility as an author to replace
%% the commands and values with those provided to you when you
%% complete the rights form.
\setcopyright{acmcopyright}
\copyrightyear{2018}
\acmYear{2018}
\acmDOI{XXXXXXX.XXXXXXX}

%% These commands are for a PROCEEDINGS abstract or paper.
\acmConference[Conference acronym 'XX]{Make sure to enter the correct
  conference title from your rights confirmation emai}{June 03--05,
  2018}{Woodstock, NY}
%%
%%  Uncomment \acmBooktitle if the title of the proceedings is different
%%  from ``Proceedings of ...''!
%%
%%\acmBooktitle{Woodstock '18: ACM Symposium on Neural Gaze Detection,
%%  June 03--05, 2018, Woodstock, NY}
\acmPrice{15.00}
\acmISBN{978-1-4503-XXXX-X/18/06}


%%
%% Submission ID.
%% Use this when submitting an article to a sponsored event. You'll
%% receive a unique submission ID from the organizers
%% of the event, and this ID should be used as the parameter to this command.
%%\acmSubmissionID{123-A56-BU3}

%%
%% For managing citations, it is recommended to use bibliography
%% files in BibTeX format.
%%
%% You can then either use BibTeX with the ACM-Reference-Format style,
%% or BibLaTeX with the acmnumeric or acmauthoryear sytles, that include
%% support for advanced citation of software artefact from the
%% biblatex-software package, also separately available on CTAN.
%%
%% Look at the sample-*-biblatex.tex files for templates showcasing
%% the biblatex styles.
%%

%%
%% The majority of ACM publications use numbered citations and
%% references.  The command \citestyle{authoryear} switches to the
%% "author year" style.
%%
%% If you are preparing content for an event
%% sponsored by ACM SIGGRAPH, you must use the "author year" style of
%% citations and references.
%% Uncommenting
%% the next command will enable that style.
%%\citestyle{acmauthoryear}


%%
%% end of the preamble, start of the body of the document source.
\begin{document}

%%
%% The "title" command has an optional parameter,
%% allowing the author to define a "short title" to be used in page headers.
\title{Uncertainty Quantification on Graph Learning: A Survey}

%%
%% The "author" command and its associated commands are used to define
%% the authors and their affiliations.
%% Of note is the shared affiliation of the first two authors, and the
%% "authornote" and "authornotemark" commands
%% used to denote shared contribution to the research.
%\author{Ben Trovato}
%\authornote{Both authors contributed equally to this research.}
%\email{trovato@corporation.com}
%\orcid{1234-5678-9012}
%\author{G.K.M. Tobin}
%\authornotemark[1]
%\email{webmaster@marysville-ohio.com}
%\affiliation{%
%  \institution{Institute for Clarity in Documentation}
%  \streetaddress{P.O. Box 1212}
%  \city{Dublin}
%  \state{Ohio}
%  \country{USA}
%  \postcode{43017-6221}
%}

%\author{Lars Th{\o}rv{\"a}ld}
%\affiliation{%
%  \institution{The Th{\o}rv{\"a}ld Group}
%  \streetaddress{1 Th{\o}rv{\"a}ld Circle}
%  \city{Hekla}
%  \country{Iceland}}
%\email{larst@affiliation.org}

%\author{Valerie B\'eranger}
%\affiliation{%
%  \institution{Inria Paris-Rocquencourt}
%  \city{Rocquencourt}
%  \country{France}
%}
%
%\author{Aparna Patel}
%\affiliation{%
% \institution{Rajiv Gandhi University}
% \streetaddress{Rono-Hills}
% \city{Doimukh}
% \state{Arunachal Pradesh}
% \country{India}}
%
%\author{Huifen Chan}
%\affiliation{%
%  \institution{Tsinghua University}
%  \streetaddress{30 Shuangqing Rd}
%  \city{Haidian Qu}
%  \state{Beijing Shi}
%  \country{China}}
%
%\author{Charles Palmer}
%\affiliation{%
%  \institution{Palmer Research Laboratories}
%  \streetaddress{8600 Datapoint Drive}
%  \city{San Antonio}
%  \state{Texas}
%  \country{USA}
%  \postcode{78229}}
%\email{cpalmer@prl.com}
%
%\author{John Smith}
%\affiliation{%
%  \institution{The Th{\o}rv{\"a}ld Group}
%  \streetaddress{1 Th{\o}rv{\"a}ld Circle}
%  \city{Hekla}
%  \country{Iceland}}
%\email{jsmith@affiliation.org}
%
%\author{Julius P. Kumquat}
%\affiliation{%
%  \institution{The Kumquat Consortium}
%  \city{New York}
%  \country{USA}}
%\email{jpkumquat@consortium.net}

%%
%% By default, the full list of authors will be used in the page
%% headers. Often, this list is too long, and will overlap
%% other information printed in the page headers. This command allows
%% the author to define a more concise list
%% of authors' names for this purpose.
\renewcommand{\shortauthors}{Trovato et al.}

%%
%% The abstract is a short summary of the work to be presented in the
%% article.
\begin{abstract}

\end{abstract}

%%
%% The code below is generated by the tool at http://dl.acm.org/ccs.cfm.
%% Please copy and paste the code instead of the example below.
%%
\begin{CCSXML}
<ccs2012>
 <concept>
  <concept_id>00000000.0000000.0000000</concept_id>
  <concept_desc>Do Not Use This Code, Generate the Correct Terms for Your Paper</concept_desc>
  <concept_significance>500</concept_significance>
 </concept>
 <concept>
  <concept_id>00000000.00000000.00000000</concept_id>
  <concept_desc>Do Not Use This Code, Generate the Correct Terms for Your Paper</concept_desc>
  <concept_significance>300</concept_significance>
 </concept>
 <concept>
  <concept_id>00000000.00000000.00000000</concept_id>
  <concept_desc>Do Not Use This Code, Generate the Correct Terms for Your Paper</concept_desc>
  <concept_significance>100</concept_significance>
 </concept>
 <concept>
  <concept_id>00000000.00000000.00000000</concept_id>
  <concept_desc>Do Not Use This Code, Generate the Correct Terms for Your Paper</concept_desc>
  <concept_significance>100</concept_significance>
 </concept>
</ccs2012>
\end{CCSXML}

\ccsdesc[500]{Do Not Use This Code~Generate the Correct Terms for Your Paper}
\ccsdesc[300]{Do Not Use This Code~Generate the Correct Terms for Your Paper}
\ccsdesc{Do Not Use This Code~Generate the Correct Terms for Your Paper}
\ccsdesc[100]{Do Not Use This Code~Generate the Correct Terms for Your Paper}

%%
%% Keywords. The author(s) should pick words that accurately describe
%% the work being presented. Separate the keywords with commas.
\keywords{Do, Not, Us, This, Code, Put, the, Correct, Terms, for,
  Your, Paper}

\received{20 February 2007}
\received[revised]{12 March 2009}
\received[accepted]{5 June 2009}

%%
%% This command processes the author and affiliation and title
%% information and builds the first part of the formatted document.
\maketitle

\section{Introduction}
\section{Preliminaries}
\label{sec:preliminaries}
In this section, we briefly introduce two key families of models for graphs, the Probabilistic Graphical Model and the Graph Neural Network.

\subsection{Probabilistic Graphical Model}
A probabilistic graphical model (PGM) \cite{bishop2006pattern,jordan2004graphical} is widely used to represent the probability distribution of random variables, whose relations can be depicted in a graph.
PGMs aim to minimize the cost of establishing compatible dependency relationships among all the variables. 
There are two distinct categories of PGMs, depending on whether the graph's edges are directed or undirected. 
Specifically, a Bayesian Network is employed for modeling a directed graph, where edges represent causality. 
Conversely, a Markov Random Field (MRF) is utilized to model an undirected graph, where edges signify the correlation between nodes. 
In an MRF, each node is conditionally independent of all other nodes, except for its immediate (e.g., 1-hop) neighbors.
% In this paper, we will focus on MRF.

Formally, 
we denote a graph $G=(V,E)$ consisting of a set of $n$ random variables $V=\{ X_1, \dots, X_n \}$ and $E$ implies the relationship between any two variables. 
Each variable $X_i \in V$ may take values in $\{1,\dots, C\}$ where $C$ is the number of classes.
% given a set of $n$ random variables $V=\{ X_1, \dots, X_n \}$, each taking values in $\{1,\dots, C\}$ where $C$ is the number of classes, 
An MRF factories the joint distribution $P(V)$ as 
\begin{equation}
\label{eq:pgm_joint}
    P(V)=\frac{1}{Z}\prod_{X_i\in V} \phi(X_i=x_i) \prod_{X_j\in \mathcal{N}(X_i)}\psi(X_i=x_i, X_j=x_j),
\end{equation}
where $Z$ normalizes the product to a probability distribution.
$\mathcal{N}(X_i)$ represents the neighbors of $X_i$ within the graph $G$.
$\phi(X_i=x_i)$ denotes the prior distribution of $X_i$ taking on the value $x_i$, independent of other variables in the graph.
The compatibility $\psi(X_i=x_i, X_j=x_j)$ encodes the likelihood of the pair
$(X_i, X_j)$ jointly taking the value $(x_i, x_j)$,
and capture the dependencies between variables.
The marginal distribution $b(X_i)$, also known as belief, for any $X_i \in V$ is naturally given by
\begin{equation}
\label{eq:pgm_margin}
    b(X_i=x_i) = \sum_{X_1} \dots \sum_{X_{i-1}} \sum_{X_{i+1}} \dots \sum_{X_n} P(X_1, X_2, \dots, X_i=x_i, \dots, X_n).
\end{equation}

However, computing $b(X_i=x_i)$ for any node $X_i \in V$ by Eq. (\ref{eq:pgm_margin}) is exponential complexity in the worst case. 
We show that Belief Propagation (BP) \cite{bishop2006pattern} infers the beliefs much more efficiently. 
First, the message $m_{i\to j}(X_j=x_j)$ from $X_i$ to $X_j$ is defined by
\begin{equation}
\label{eq:pgm_sum_prod}
\frac{1}{Z_j}\sum_{X_i=x_i}
\left[\psi(X_i=x_i, X_j=x_j)\phi(X_i=x_i) \prod_{k\in \mathcal{N}(X_i)\setminus \{X_j\}} m_{k\to i}(X_i=x_i)\right],
\end{equation}
where $Z_j$ is a normalization factor so that $m_{i\to j}$ is a probability distribution of $X_j$.
The messages in both directions on all edges $(X_i, X_j) \in E$ will be updated iteratively. 
As the message updating converges (guaranteed when $G$ is acyclic~\cite{pearl1988probabilistic}).
The belief $b(X_i=x_i)$ is given by
\begin{equation}
\label{eq:pgm_belief}
    b(X_i=x_i) \propto \phi(X_i=x_i)\prod_{X_j\in \mathcal{N}(X_i)} m_{j\to i}(X_i=x_i).
\end{equation}
For simplicity, we will omit $X_i$ in $\phi(X_i=x_i)$ and use $\phi(x_i)$, and similarly for $\psi(x_i,x_j)$, $m_{j\to i}(x_i)$ and $b(x_i)$ if there is no ambiguity.

\subsection{Graph Neural Networks}
Graph Neural Network (GNN) is a deep learning-based method that 
\section{Sources of Uncertainty}
\section{Methods for Handling Uncertainty}

%In the field of graph learning, Probabilistic Graphical Models (PGM) and Graph Neural Networks (GNNs) are two predominant methods for processing graph data. 
%This section focuses on exploring various Uncertainty Quantification (UQ) techniques that are specifically designed to measure and quantify the predictive uncertainties of these models. 
%In this realm, the targets of UQ can vary widely, including but not limited to node or graph classification predictions, and quantities of interest (QoIs) relevant to specific learning tasks. 
%\subsection{Uncertainty Quantification in Probabilistic Graphical Models}

%Applying PGMs to graph data involves probabilistic reasoning and learning on graph-structured data. 
%In this context, nodes and edges in the graph represent random variables and the probabilistic relationships between these variables, respectively. 
%This probabilistic modeling of the graph enables the use of PGM-specific inference and learning algorithms for various tasks like node classification, link prediction, and more. 
%Simultaneously, the UQ methods for PGMs can naturally provide uncertainty estimates for predictions related to these graph learning tasks. 

%\subsubsection{UQ of Marginal Probabilities for Variable Node} 

%\subsubsection{UQ of QoIs}
%\subsection{Uncertainty Quantification in Graph Neural Networks}


%\section{Acknowledgments}

\bibliographystyle{ACM-Reference-Format}
\bibliography{sample-base}

\appendix

\end{document}
\endinput
%%
%% End of file `main.tex'.
