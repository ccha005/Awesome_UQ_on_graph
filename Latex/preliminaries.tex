\section{Preliminaries}
\label{sec:preliminaries}
In this section, we briefly introduce two key families of models for graphs, the Probabilistic Graphical Model and the Graph Neural Network.

\subsection{Probabilistic Graphical Model}
A probabilistic graphical model (PGM) \cite{bishop2006pattern,jordan2004graphical} is widely used to represent the probability distribution of random variables, whose relations can be depicted in a graph.
PGMs aim to minimize the cost of establishing compatible dependency relationships among all the variables. 
There are two distinct categories of PGMs, depending on whether the graph's edges are directed or undirected. 
Specifically, a Bayesian Network is employed for modeling a directed graph, where edges represent causality. 
Conversely, a Markov Random Field (MRF) is utilized to model an undirected graph, where edges signify the correlation between nodes. 
In an MRF, each node is conditionally independent of all other nodes, except for its immediate (e.g., 1-hop) neighbors.
% In this paper, we will focus on MRF.

Formally, 
we denote a graph $G=(V,E)$ consisting of a set of $n$ random variables $V=\{ X_1, \dots, X_n \}$ and $E$ implies the relationship between any two variables. 
Each variable $X_i \in V$ may take values in $\{1,\dots, K\}$ where $K$ is the number of classes.
% given a set of $n$ random variables $V=\{ X_1, \dots, X_n \}$, each taking values in $\{1,\dots, C\}$ where $C$ is the number of classes, 
An MRF factories the joint distribution $P(V)$ as 
\begin{equation}
\label{eq:pgm_joint}
    P(V)=\frac{1}{Z}\prod_{X_i\in V} \phi(X_i=x_i) \prod_{X_j\in \mathcal{N}(X_i)}\psi(X_i=x_i, X_j=x_j),
\end{equation}
where $Z$ normalizes the product to a probability distribution.
$\mathcal{N}(X_i)$ represents the neighbors of $X_i$ within the graph $G$.
$\phi(X_i=x_i)$ denotes the prior distribution of $X_i$ taking on the value $x_i$, independent of other variables in the graph.
The compatibility $\psi(X_i=x_i, X_j=x_j)$ encodes the likelihood of the pair
$(X_i, X_j)$ jointly taking the value $(x_i, x_j)$,
and capture the dependencies between variables.
The marginal distribution $b(X_i)$, also known as belief, for any $X_i \in V$ is naturally given by
\begin{equation}
\label{eq:pgm_margin}
    b(X_i=x_i) = \sum_{X_1} \dots \sum_{X_{i-1}} \sum_{X_{i+1}} \dots \sum_{X_n} P(X_1, X_2, \dots, X_i=x_i, \dots, X_n).
\end{equation}

However, computing $b(X_i=x_i)$ for any node $X_i \in V$ by Eq. (\ref{eq:pgm_margin}) is exponential complexity in the worst case. 
We show that Belief Propagation (BP) \cite{bishop2006pattern} infers the beliefs much more efficiently. 
First, the message $m_{i\to j}(X_j=x_j)$ from $X_i$ to $X_j$ is defined by
\begin{equation}
\label{eq:pgm_sum_prod}
\frac{1}{Z_j}\sum_{X_i=x_i}
\left[\psi(X_i=x_i, X_j=x_j)\phi(X_i=x_i) \prod_{k\in \mathcal{N}(X_i)\setminus \{X_j\}} m_{k\to i}(X_i=x_i)\right],
\end{equation}
where $Z_j$ is a normalization factor so that $m_{i\to j}$ is a probability distribution of $X_j$.
The messages in both directions on all edges $(X_i, X_j) \in E$ will be updated iteratively. 
As the message updating converges (guaranteed when $G$ is acyclic~\cite{pearl1988probabilistic}).
The belief $b(X_i=x_i)$ is given by
\begin{equation}
\label{eq:pgm_belief}
    b(X_i=x_i) \propto \phi(X_i=x_i)\prod_{X_j\in \mathcal{N}(X_i)} m_{j\to i}(X_i=x_i).
\end{equation}
For simplicity, we will omit $X_i$ in $\phi(X_i=x_i)$ and use $\phi(x_i)$, and similarly for $\psi(x_i,x_j)$, $m_{j\to i}(x_i)$ and $b(x_i)$ if there is no ambiguity.

\subsection{Graph Neural Networks}
Graph Neural Network (GNN) is a deep learning-based method that works with graph data. 
Unlike traditional neural networks that operate on grid-like data (such as images or sequences),  
GNN could learn from complex relationships and dependencies in graphs.

Formally, 
given a graph $G=(V, E)$, $V$ is the set of nodes and $E$ is the set of edges connecting the nodes. 
Let $\mathcal{N}(X_i)$ be the set of neighbors of node $X_i\in V$, $i=1,\dots, |V|$.
Assume that a GNN of $L$ layers is trained to predict class distributions of the nodes on the graph $G$.
On the $l$-th layer, $l=1,\dots, L$, 
GNN calculates $\mathbf{h}_i^{(l)}$ for node $X_i$ considering messages sent from its neighborhood $\mathcal{N}(X_i)$.
Formally,
\begin{eqnarray}
% \mathbf{m}^{(l)}_{ji} & = & \textnormal{MSG}\left(\mathbf{h}^{(l-1)}_j,\mathbf{h}^{(l-1)}_i\right),\\
\mathbf{a}^{(l)}_i & = & \textnormal{AGG}\left(\left\{\mathbf{m}^{(l)}_{ji}|X_j \in \mathcal{N}(X_i)\right\}\right),\\
\mathbf{h}^{(l)}_i & = & \textnormal{UPDATE}\left(\mathbf{a}^{(l)}_i, \mathbf{h}^{(l-1)}_i \right).
\end{eqnarray}
% The MSG function computes the message sent from $X_j$ to $X_i$ (e.g., $\mathbf{m}_{ji}^{(l)}=\mathbf{h}_{j}^{(l-1)}$).
The AGG function aggregates the messages sent from all $X_j\in \mathcal{N}(X_i)$ to $X_i$, which can be the element-wise average, maximum of the messages, or a parametric layer \cite{xu2018powerful}.
The UPDATE function combines representations from the $l-1$ and the current $l$ layers.
For example, it is defined as $W\cdot \left[ \mathbf{h}_i^{(l-1)}, \mathbf{a}_i^{(l)} \right]$ with trainable matrix $W$ in GraphSAGE \cite{hamilton2017inductive}.
$\mathbf{h}^{(0)}_i$ is the input node feature for node $X_i$, 
and the output of the GNN is $\mathbf{h}_i^{(L)}$, which can be softmaxed to a class distribution $\pi(X_i)$.
The parameters of GNN, $\boldsymbol{\theta} = \{ \theta^{(l)}$, $l=1,\dots,L$ \},
are trained in an end-to-end style using labeled nodes on $G$.
