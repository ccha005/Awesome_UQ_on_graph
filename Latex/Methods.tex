\section{Methods for Handling Uncertainty}

%In the field of graph learning, Probabilistic Graphical Models (PGM) and Graph Neural Networks (GNNs) are two predominant methods for processing graph data. 
%This section focuses on exploring various Uncertainty Quantification (UQ) techniques that are specifically designed to measure and quantify the predictive uncertainties of these models. 
%In this realm, the targets of UQ can vary widely, including but not limited to node or graph classification predictions, and quantities of interest (QoIs) relevant to specific learning tasks. 
%\subsection{Uncertainty Quantification in Probabilistic Graphical Models}

%Applying PGMs to graph data involves probabilistic reasoning and learning on graph-structured data. 
%In this context, nodes and edges in the graph represent random variables and the probabilistic relationships between these variables, respectively. 
%This probabilistic modeling of the graph enables the use of PGM-specific inference and learning algorithms for various tasks like node classification, link prediction, and more. 
%Simultaneously, the UQ methods for PGMs can naturally provide uncertainty estimates for predictions related to these graph learning tasks. 

%\subsubsection{UQ of Marginal Probabilities for Variable Node} 

%\subsubsection{UQ of QoIs}
%\subsection{Uncertainty Quantification in Graph Neural Networks}
