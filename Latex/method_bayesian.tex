% \section{Methods for Handling Uncertainty}
\subsection{Bayesian Methods}
\subsubsection{Bayesian Methods Overview}
\subsubsection{Bayesian Methods for Graphs}

Bayesian methods for semi-supervised node classification on graphs focus on obtaining the posterior probability distribution of nodes. One route lies in assigning a prior probability distribution to each node, then updating it with evidence propagated from other nodes to obtain the posterior distribution. Yamaguchi et al. \cite{yamaguchi2015socnl} assumed the label is a categorical random variable as $P(\hat{y}_i=k|\boldsymbol{\theta})$, where $\boldsymbol{\theta}$
 is the parameter of the categorical distribution. 
 Based on the smoothness hypothesis, they believed that a neighbor of node $i$ shares the same parameter $\boldsymbol{\theta}$ as $i$. 
 This leads to the multinomial likelihood function of labels of neighbors of $i$ $P(\hat{N}_i|\boldsymbol{\theta})\propto\Pi_{k=1}^K\theta_k^{n_{ik}}$ and the conjugate Dirichlet prior $P(\boldsymbol{\theta})\propto\Pi_{k=1}^K\theta_k^{\alpha_k-1}$
 , where $n_{ik}$ is the number of i’s neighbors whose label is $k$, and $\boldsymbol{\alpha}=(\alpha_1,\alpha_2,\cdots,\alpha_K)^T$ is the parameter of Dirichlet distribution. 
 Combining these, the posterior distribution of $\boldsymbol{\theta}$ is $P(\boldsymbol{\theta}|\hat{N}_i)\propto\Pi_{k=1}^K\theta_k^{\alpha_k+n_{ik}-1}$. Beyond the smoothness hypothesis and label propagation, Eswaran et al. \cite{eswaran2017power} used compatibility matrices to represent the strength of connections between nodes and propagates multinomial messages to support both homophily and heterophily network effects. 
 To handle high-dimensional features of nodes, Stadler et al. \cite{stadler2021graph} employed MLP encoding and computes node-level pseudo-counts, then propagates them via PPR-based message passing. All three methods utilize Dirichlet priors and posteriors, with the main difference being in the form of multinomial distribution evidence propagated.

 Another route lies in transferring Bayesian neural networks to GNNs. Based on the idea of treating parameters as probability distributions, Munikoti et al. \cite{munikoti2023general} proposed a Bayesian graph neural network framework for quantifying aleatoric and epistemic uncertainties. They derived the relationship between the mean and variance of neurons in one layer of the GNN with the previous layer to propagate aleatoric uncertainty to the model output, and obtained epistemic uncertainty through Monte Carlo dropout.

%In the field of graph learning, Probabilistic Graphical Models (PGM) and Graph Neural Networks (GNNs) are two predominant methods for processing graph data. 
%This section focuses on exploring various Uncertainty Quantification (UQ) techniques that are specifically designed to measure and quantify the predictive uncertainties of these models. 
%In this realm, the targets of UQ can vary widely, including but not limited to node or graph classification predictions, and quantities of interest (QoIs) relevant to specific learning tasks. 
%\subsection{Uncertainty Quantification in Probabilistic Graphical Models}

%Applying PGMs to graph data involves probabilistic reasoning and learning on graph-structured data. 
%In this context, nodes and edges in the graph represent random variables and the probabilistic relationships between these variables, respectively. 
%This probabilistic modeling of the graph enables the use of PGM-specific inference and learning algorithms for various tasks like node classification, link prediction, and more. 
%Simultaneously, the UQ methods for PGMs can naturally provide uncertainty estimates for predictions related to these graph learning tasks. 

%\subsubsection{UQ of Marginal Probabilities for Variable Node} 

%\subsubsection{UQ of QoIs}
%\subsection{Uncertainty Quantification in Graph Neural Networks}
